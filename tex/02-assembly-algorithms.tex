\chapter{Assembly algorithms}

\section{Assembly problem formulation}

There are several formulations of the genome assembly problem, but
most of them are not user in practice.

First of them is a variation of shortest common superstring problem, where we
almost account for the possibility of sequencing errors \cite{kececioglu1995combinatorial}. 

\begin{definition}{DNA sequence reconstruction problem 1.}
Given set of reads $\mathcal{F}$ and error rate $\varepsilon$, find
a shortest sequence $S$ such that for every $A \in \mathcal{F}$ there is a substring
$B$ of $S$ such that:
$$min(d(A,B), d(\bar{A}, B)) \leq \varepsilon |A|$$
\end{definition}

This formulation has several problems. 
Firstly, it compresses repeated regions in genome. For example take sequence
$ACGGGGGGGGGGGGTT$ and all of its substrings of length 3. Then the correct
output for our formulation is $ACGGGTT$.

Secondly, it is ambiguous. For example take sequence
$ACGGGGTATGGGGCTCGGGGAA$ and all of its substring of length 3. Then the correct
outputs for our formulation are $ACGGGGTATGGGGCTCGGGGAA$ and
$ACGGGGCTCGGGGTATGGGGAA$.

A better formulation is given by \cite{myers1995toward}. It also considers
coverage of output sequence by substring and wants it to be as uniform as possible.

We consider our reconstructed string $S$ and the layout consisting of
$F$ pairs of integers $(s_i, e_i)$, which indicates starting and ending positions
of reads in the reconstructed sequence. The layout is $\varepsilon$-valid
if for each read $A$ the edit distance between $S[s_i:e_i]$ and the read
is at most $\varepsilon |A|$.

We will now formalize the notion of uniform coverage. Lets consider a observed
distribution of read start points (the proportion of reads which start before $x$):
$$D_{obs}(x) = \frac{|\{s_i < x\}|}{F}$$

We now consider a source distribution of a sampling process $D_{src}$ (which is usually
uniform, but can be nonuniform due to some systematic errors) and define maximum
deviation between these two distributions:
$$\delta = max |D_{obs}(x) - D_{src}(x)|$$

Now we can define DNA sequence reconstruction problem in a better way:

\begin{definition}{DNA sequence reconstruction problem 2.}
Given set of reads $\mathcal{F}$ and error rate $\varepsilon$, find
a sequence $S$ and $\varepsilon$-valid layout which has minimal
maximum deviation between observed and source distrition of reads.
\end{definition}

There are still some ambiguities and problems with this formulation.
One can for example find two solutions which have same maximum deviation
but differ in one base. There might be also problem with reads which have too many errors,
which were created by contamination during sequencing process by other DNA.

\section{Assembly algorithms overview}

Almost all assembly algorithms used in practice are some form of heuristics
without well defined formulation, proof of correctness, etc.
There are well defined algorithms, for example algorithm by \cite{Medvedev2009}
which uses bidirectional flows, but they practical use is very limited (this one
assumes error-free reads).

The goal of assembly algorithms is to "glue" reads which can be unambiguously glued together.
They use efficient representation of overlaps between reads and try to resolve
ambiguous regions using paired end reads and long reads.

The good review of assembly algorithms can be found in \cite{miller2010assembly}.
They can be divided into two types by representation of overlaps they use.
The overlap-layout-consensus algorithms use overlap graph, which directly represents
overlaps between reads. Nowadays this is represented as string graph (TODO cite).
The other algorithms use deBruijn graphs, which do not work directly with reads, but
with sequences of $k$-bases ($k$-mers). The nodes in deBruijn graph represent $k$-mers
and edges represent adjacencies between $k$-mers in reads. 

\section{Overlap-layout-consensus algorithms}

\section{deBruijn graph algorithms}
