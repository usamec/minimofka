\chapter{Evaluating quality of assemblies}

After performing an assembly we can count several
interesting statistics, which can tell us how good the assembly is.

\section{Basic statistics}

\subsection{Statistics based only on assembly}

This statistics summarize the distribution of lengths of contigs.
We give a brief overview of metrics used in QUAST (\cite{Quast}), which
is standard tool for evaluation assemblies.

\begin{itemize}
\item Number of contigs (or number of contigs longer than $x$ -- where usually $x = 1000$).
\item Length of the largest contig.
\item Total length of contigs.
\item $Nx$ (where $0 < x \leq 100$) -- the largest contig length $L$, such that using contigs
of length $\geq L$ accounts for at least $x \%$ of the bases of the assembly.
\end{itemize}

Note that each of this statistics can be "gambled" so comparing assemblies based
on this statistics can be only done when we assume, that assembly software does only
reasonable operations.

\subsection{Statistics based on assembly and reference sequence}

Sometimes (especially during evalution as assembly algorithms) we have access to true sequence
and we can compute various statistics which tell us how many errors are in our assembly.
Calculating these statistics usually starts with aligning assembly and reference genome, which
gives us information in the form:
"Substring of assembly starting at position $a$, ending at position $b$ can be mapped
to the substring of reference genomes starting at $c$, ending at $d$ with $e$ edits."
We called aligned substrings blocks. 

\begin{itemize}
\item No. of missasemblies -- missassembly is usually defined as a positions where
positions of block aligned on the left and block aligned on the right differ in reference
by more that $1000$ bases or this block are on opposite strands of in different chromosomes.
\item No. of missassembled contigs -- nubmer of contigs which contain missassembly.
\item No. of mismatches and indels -- these statistics are interesting in one base differences,
either substitutions or insertions and deletions.
\item $NAx$ -- similar to $Nx$, but before computing this statistic we break
contigs at missassemblies.
\end{itemize}

\section{REAPR}

\section{Probability models}

