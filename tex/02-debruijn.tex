\chapter{De Bruijn graphs}

De Bruijn graph (\cite{de1946combinatorial}) is a structure for representing
overlaps between sequences of symbols.

De Bruijn graph is directed graph. 
For given $k$, vertices of De Bruijn graph represent sequences of length $k$
($k$-mers). The edges of De Bruijn graph represent possible overlaps, i. e. 
if have have an edge $(u, v)$ and vertex $u$ represents $k$-mer $a_1, a_2, \dots, a_k$, then
vertex $v$ represents $k$-mer $a_2, a_3, \dots, a_k, x$ and $k$-mer in vertex $v$ can
follow $k$-mer in vertex $u$.

\section{De Bruijn graphs for assembly}

We can use De Bruijn graphs for DNA sequence assembly in a following way
(\cite{pevzner2001eulerian}, \cite{Velvet}):

We are given odd number $k$.
Each vertex represents one $k$-mer. For each vertex $v$ there is a twin vertex $v^{'}$ which
represents reverse complemented $k$-mer (we need to do this to deal with reads
coming from reverse complement part of the DNA). Union of $v$ and $v^{'}$ is called a block.
During manipulation of De Bruijn graph any change applied to node $v$ is also symetrically
applied to its twin $v^{'}$. To ensure, that some $k$-mer is to same as its reverse complement,
$k$ must be an odd number.

Vertices $u$, $v$ in De Bruijn graph can be connected by directed arcs if the last
$k-1$ bases of $u$ are same as the first $k-1$ bases in $v$. Also note that
if there is an edge between vertices $u$, $v$ there is an inverted edge between their twins,
i.e. there is an edge from $v^{'}$ to $u^{'}$.

We construct De Bruijn graph from reads in straighforward way.
We extract every $k$-mer from each reads and its reverse complement.
That we joined by edge each pair of $k$-mer which are overlapped in some read.
Some assemblers also record some simple statistics like multiplicity of each vertex or edge. 
This construction can be done in efficient way by using hash tables, like in Velvet (\cite{Veltet}).
There are also space-efficient representations of De Bruijn graphs based on Bloom filters (\cite{minia}).

TODO obrazok

After constructing De Bruijn graph we have several options how to produce the assembly:

We can look for Eulerian path through graph. Even better approach is to look for
Eulerian superpath -- the path which visits every read, i.e. if some read
is broken into vertices $v_1, v_2, \dots, v_n$, then the Eulerian superpath
should contain $v_1, \dots, v_n$ as subpath.

\begin{definition}{Eulerian superpath problem.}
Given graph $G$ and set of paths $\mathbb{P}$ find path $P$, which contains
every path from $\mathbb{P}$ as subpath.
\end{definition}

Euler assembler (\cite{pevzner2001eulerian}) solves some instances of Eulerian superpath problem
using simple transformations which transform Eulerian superpath instance to equivalend Eulerian path instance.

Other assemblers like Velvet (\cite{Velvet}) just join vertices which can be joined
unambiguosly, i.e. they join vertices $u$, $v$ is there is an edge from $u$ to $v$ and
it is only edge comming out of $u$ and only edge comming into $v$. After this joining
each vertex represents one contig of the assembly.

\subsection{Handling sequencing errors}

Sequencing errors in reads increase complexity of the De Bruijn graph
(\cite{pevzner2001eulerian}). The common artefacts in graph are tips and bubbles
(TODO obrazok). Assemblers like Velvet (\cite{Velvet}), ABySS (\cite{Abyss}) handle
errors directly in De Bruijn graph by detecting this artefacts and removing them.

The naive approch to fix errors in graph can be to remove all $k$-mers with low coverage.
This can be problematic since some sequencing technologies do not produce reads with uniform coverage
and we will remove good low coverage regions and we can cause the graph to be disconnected.

\paragraph{Removing tips.} A tip is chain of vertices disconnected at one end. 
Tips usually represent some local errors in the reads. Their removal is pretty straighforward
and removing them does not destroy connectivity in the graph. But sometimes the tips
might represent correct sequence which is interupted by gap in the coverage.
Velvet (\cite{Velvet}) removes tips, which are at most $2k$ bases long and the edge starting
the tip has lower multiplicity that some other edge in the branching vertex.

\paragraph{Removing bubbles.} Two paths which start and end at the same nodes (form a bubble)
and contain similar sequences might be considered redundant.
For example these bubbles might be caused by small error in the middle of the read.
Velvet find similar path using following algorithm:
It starts breath first search like algorithm from arbitrary vertex, but considers higher converage
arcs as shorter. Whenever it encounters previous visited vertex $v$ it tries to find closest common ancestor
of current path and shortest path $v$. Then it extract seqeunce belonging to two paths and if this two
paths is judged as similar enough the longer path is merged into shorter one. 
 
\bigskip

Other option is to use tools like QUAKE (\cite{Quake}), which try to correct reads
without assembling them. This tools usualy consider $k$-mers with low abundance
as errornous and try to correct them using few simple edits in reads.


\subsection{Achieving longer contigs}

When using De Bruijn graphs we discarded a lot of continuity information from reads.
For example if we have reads of length $100$ bases, we can resolve assembly around repeats
of length up $100$ bases. But after we construct De Bruijn graph with $k = 33$ we are only
able to resolve assembly around repeats of length up to $33$ bases. Also De Bruijn graphs
do not directly support handling paired reads.

For example ABySS (\cite{Abyss}) uses paired reads in following way:
After initial error correction in De Bruijn graph in align paired reads
to reasonably long vertices in De Bruijn graph. Two vertices are considered linked
if there at least $p$ read pairs (usually $p=5$) which join the vertices.
We denote $P_i$ the set of vertices which are linked to vertex $v_i$.
Then we perform search for single unique path from $v_i$ through De Bruijn graph which visits
each vertex from $P_i$. We usualy limit this search by estimated distance from $v_i$ to each element
from $P_i$. This process is performed for each vertex and final consistent paths are stitched to final assembly.

\section{Pair De Bruijn graphs}

De Bruijn graphs do not incorporate any information from paired reads.
Assemblers mentioned above usually use information from paired reads only in heuristics postprocessing.
These heuristics usually work well is simple cases, but fail when assembly contains several repeated regions after each other.

Interesting concept are paired De Bruijn graphs (\cite{Paired}) which try to incorporate paired read
information directly info De Bruijn graph. In the following we will present construction of 
paired De Bruijn graph. For simplicity we will ignore reverse complements a problems associated with it.

\paragraph{Paired De Bruijn graphs with exact insert sizes.}
Let $k$-bimer be pair of two $k$-mer $(a, b)$. Each vertex represents one $k$-bimer and edges represent
consecutive $k$-bimers similarly as in ordinary De Bruijn graph. 
From each pair of reads we will produce $k$-bimers in straighforward way:
$i$-th $k$-bimer will be created using $i$-th $k$-mer from the first read and $i$-th $k$-mer from the second read.
Edges in graph are given by consecutive pairs of $k$-mers from reads.

The assembly is then created by walking some path through graph where we read one part of $k$-bimers.

\paragraph{Paired De Bruijn graphs with approximate insert sizes.}
Lets assume that insert sizes are from the interval $d \pm \Delta$.
We start by constructing paired De Bruijn graph described in previous paragraph.
The key insight is that if two $k$-bimers $(a, b)$ and $(a, b^{'})$ come from
same location of $a$ then the distance of $b$ and $b^{'}$ in ordinary De Bruijn graph is at most $2\Delta$.

In the next step we will merge all vertices $(a, b)$, $(a, b^{'})$ if there is a path
between $b$ and $b^{'}$ in ordinary De Bruijn graph with length at most $2\Delta$.

\bigskip
Like in the case of ordinary De Bruijn graph we need to deal with errors in reads
and also with insert size outliers. This approach still cannot handle multiple
read libraries with various insert sizes and combination of paired libraries and long reads.
One implementation of this approach is SPAdes (\cite{Spades}).



\section{Correction long reads with high number of errors}
