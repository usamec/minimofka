\chapter{Introduction}

\section{Problem of sequence assembly}

DNA sequence of an organism is a string over the four letter alphabet A, C, G, T
(with usual length between millions to billions characters).
Current technologies cannot read the whole sequence at once but produce many
substrings of sequence called fragments or reads (whose positions usually overlaps).
In most cases locations of substrings are sampled using uniform distribution.

The goal of sequence assembly is to reconstruct original string.

One might thought that shortest common superstring (\cite{maier1977note}) 
is a good formalization of the problem above. But consider following DNA sequence:

$$AA\textcolor{red}{CGTA}\textcolor{green}{CGTA}\textcolor{blue}{CGTA}GG$$

If our reads have length 4 and come from all possible positions the shortest
common superstring would be:
$$AA\textcolor{red}{CGTA}\textcolor{green}{CGTA}GG$$

In this case we lost one repetitions of repeated sequence. If we assume uniform
sampling from original sequence we might estimate the number of repeated substring
by using coverage information (how many reads cover one position).

But there are also sequences for which we cannot reconstruct the order of elements in
original sequence, consider the following:

$$AA\textcolor{red}{CTCT}\textcolor{green}{GG}\textcolor{red}{CTCT}\textcolor{blue}{CC}\textcolor{red}{CTCT}TT$$

In this case if have reads of length 4 we cannot distinguish between sequence above and
sequence:

$$AA\textcolor{red}{CTCT}\textcolor{blue}{CC}\textcolor{red}{CTCT}\textcolor{green}{GG}\textcolor{red}{CTCT}TT$$

Due to this, the goal of the sequence assembly is to reconstruct as long as possible
unambigous parts of the DNA sequence.

\section{Real-life complications}

In practice there are several complications which make sequence assembly even harder.
In following text we mention the major ones.

\subsection{Reverse complement}

Each letter (base) from DNA alphabet has the complementary base. This can
be viewed as homomorphism $h(\cdot)$, where: $h(A) = T, h(C) = G, h(G) = C, h(T) = A$.
The real DNA is composed from two strings: $S$ and $h(S^R)$ -- the reverse complement.
During sequencing process the bases can come from both sides of the DNA and
we do not have information about the side.

\subsection{Errors in reads}

In practice the chemical process of sequencing DNA also produces errors.
Sometimes there are small errors in reads -- i.e. substitutions, insertions
and deletions. The amount of these errors depends on specific sequencing technology
(also some technologies have higher amount of substitutions while other
have high amount of insertions).

There are also reads which do not belong to sequenced genome, but are result
of some contamination.

\subsection{Paired-end reads}

Some technologies start reading fragments from one end and lose accuracy after
reading few hundred bases. To capture information which contains longer part of the
sequence some technologies produce longer fragments from which they readtens to hundreds
of bases from both sides and do not read bases from middle (since there will be
too many errors). The result of this process is a pair of reads for which we know approximate
distance in the DNA sequence.

\subsection{Sequencing technologies}

We summarize currently used sequencing technologies in \ref{tab:techs}.
This presents additional chalenge for assembling algorithms since they
cannot be tailored only for one technology, but should work with combination
of data from many technologies.

\begin{table}[h]
\centering
\begin{tabular}{|c|c|c|c|}
\hline
Technology & Read length & Error rate & Paired end reads \\\hline
Illumina & 50 - 300 & 2\% & Yes\\\hline
454 & 700 & $0.1\%$ & Yes \\\hline
PacBio & 2000 - 20000 & $14\%$ & No \\\hline
\end{tabular}
\caption{Overview of current sequencing technologies}
\label{tab:techs}
\end{table}

\section{Overview of known solutions}


