\chapter{Introduction}

\section{Problem of sequence assembly}

DNA sequence of an organism is a string over the four letter alphabet A, C, G, T
(with usual length between millions to billions characters).
Current technologies cannot read the whole sequence at once but produce many
substrings of sequence called fragments (whose positions usually overlaps).
In most cases locations of substrings are sampled using uniform distribution.

The goal of sequence assembly is to reconstruct original string.

One might thought that shortest common superstring (\cite{maier1977note}) 
is a good formalization of the problem above. But consider following DNA sequence:

$$AA\textcolor{red}{CGTA}\textcolor{green}{CGTA}\textcolor{blue}{CGTA}GG$$

If our fragments have length 4 and come from all possible positions the shortest
common superstring would be:
$$AA\textcolor{red}{CGTA}\textcolor{green}{CGTA}GG$$

In this case we lost one repetitions of repeated sequence. If we assume uniform
sampling from original sequence we might estimate the number of repeated substring
by using coverage information (how many fragments cover one position).

But there are also sequences for which we cannot reconstruct the order of elements in
original sequence, consider the following:

$$AA\textcolor{red}{CTCT}\textcolor{green}{GG}\textcolor{red}{CTCT}\textcolor{blue}{CC}\textcolor{red}{CTCT}TT$$

In this case if have reads of length 4 we cannot distinguish between sequence above and
sequence:

$$AA\textcolor{red}{CTCT}\textcolor{blue}{CC}\textcolor{red}{CTCT}\textcolor{green}{GG}\textcolor{red}{CTCT}TT$$


\section{Real-life complications}

\subsection{Sequencing technologies}

\subsection{Paired-end reads}

\subsection{Errors in reads}

\subsection{Reverse complement}

\section{Overview of known solutions}


